\documentclass[12pt,a4paper,titlepage]{article}
%\usepackage[doublespacing]{setspace}
\usepackage[utf8]{inputenc}         %This is used for ASKII digits.
\usepackage{amsmath, amssymb}       %This is for mathematic statements,check"amath.colorado.edu/documentation/LaTex/Symbols.pdf"
\usepackage{amsfonts,mathrsfs}      %fonts~~
\usepackage{graphicx}               %This is for graph inserting.
\usepackage{paralist}               %Give me compact lists!!!!! 'compactitem' 'compactenum' 'compactdesc'
%\usepackage{bm}                     %various kinds of bold texts.
\usepackage[top =2.54cm, bottom =2.54cm, left =3.18cm, right =3.18 cm]{geometry}
\usepackage{indentfirst}            %Indent the first letter
\usepackage{fancyhdr}               %Below is the head of the article.
\pagestyle{fancy}
\usepackage{lastpage}
\lhead{Team \# 33131}
\rhead{Page \thepage{} of \pageref{LastPage}}
\cfoot{}
%\numberwithin{equation}{section}    %Numbering of the equations will be based on sections.
%\usepackage{amsthm}
%\theoremstyle{definition}
%\newtheorem{def}{Definition}  %Below set the theorem system.
%\theoremstyle{plain}
%\newtheorem{thm}[law]{Thm}
%\theoremstyle{remark}
%\newtheorem*{remark}{Remark}
\newcommand{\boldit}[1]{\textbf{\textit{#1}}}
\usepackage{float}
\usepackage{paralist}
\usepackage{url}

\begin{document}

\title{Human Capital Model} \date{\today{}}
\maketitle

\tableofcontents

\newpage

\section{Introduction}
\label{sec:introduction}

Considering the shortage of the talent, it's essential for companys to
retain good people and make them well-trained. However, current
situation is not satisfactory while many talents always tend to get a
good job via job-hopping, causing orgnizational churn in employees who
are closely connected to them. In order to simulate this process and
improve it, we build a human capital model based on Social Network
Analysis and Markov process.

\subsection{Restatement of the Problem}
\label{sec:restatement-of-the-problem}

\subsection{Literature Review}
\label{sec:literature-review}

\begin{itemize}
\item
\item
\item
\end{itemize}

\section{Terminology}
\label{sec:terminology}

\begin{itemize}
\item
\item
\item
\end{itemize}

\section{Assumptions and Justifications}
\label{sec:assumptions-and-justifications}

\begin{itemize}
\item \textbf{assumption 1} If an employee has probability to
  promote, he won't churn.

The possibility of the unforeseen accidents, which could force an
employee to leave his position,is neglected.Human nature, an employee
will stay at his position to chase for higher level.

\item \textbf{assumption 2} For a vacancy,if there exists an
  employee measures up to it already,ICM won't recruit for it.

Since recruiting good people is difficult, time consuming and
expensive according to issue 5,it is wasteful to recruit for a
position if an employee can promote to it.

\item \textbf{assumption 3} Demotion won't occur.

\item \textbf{assumption 4} Administrative clerk won't promote or be
  transferred.

\item \textbf{assumption 5} For the promotion probability and
  organization change, the other factors effects the churn probability
  is invariable.

Though churn derives from varieties of reasons and they are actually
lacking of known conditions and data to estimate them, we have to
regard it as stable in our model.

\item \textbf{assumption 6} Each division or office have at least one
  middle manager or senior manager.

\end{itemize}

% the main section of the paper
\section{Human Capital Model}
\label{sec:human-capital-model}

\subsection{Model Overview}
\label{sec:model-overview}

Most research for human capital can be classified as either
microscopic and macroscopic. Since either macroscopic or microscopic
methods are difficult to solve our problem perfectly, we approach the
problem with the combination of macroscopic and microscopic methods.

To measure the ability of each person, we use Quantitative Management
Performance. Via this measurement, we can classify differient kinds of
employees which will influence the promotion process.

To build the employee network and analyze its properties, we employ
the social network analysis (SNA) technique. In our case, the
employees are viewed as nodes and relationships as links among
them. So, We can simulate the complex relationship via this network.

Definitions of symbols employed in this paper are listed in
\textbf{Table 1}.
\begin{table}
\begin{tabular}{|l|l|}
  Variable & Description \\
  \hline
  $i$            &Index of an employee \\
  $L_{i,t}$          &The level of $i$ \\
  $s_{ij,t}$        &The relation strength from $i$ to $j$ \\
  $a_{ij,t}$        &Represent the influence caused by relationship
                    between superior and subordinate \\
  $f_{ij,t}$        &Represent the influence caused by relationship
                    between person with many friends and person with
                    few friends \\
  $c_{i,t}$        &Clustering Coefficient of $i$
\end{tabular}
\end{table}

\subsection{Human Performance Model}
\label{sec:human-model}

In this part, we buid a people model to evaluate an employee in four
aspects, in terms of \textbf{Quantitative Management Performance} ---
work achievement, work ability, work attitude and potential. These
four aspects are supposed to be quantized according to annual
evaluation based on performance as judged by the supervisor and we
take these independent variables as $A_{ac_i}$, $A_{ab_1}$, $A_{at_i}$
and $A_{po_i}$ for each employee $i$. For each of the four parameters,
it goes from 0 to 1. The statues are used for calculating the
probability that employee can promote. Meanwhile, they influence
leaving probability and team cohesiveness as well.

It is obvious that some of the parameters are somehow more important
than others. So in an effort to make our model more accurate and
reliable,we introduce a weighted index of deviation $AD_i$, with
\begin{equation}
  A_i=w_{ac} \cdot A_{ac_i} + w_{ab} \cdot A_{ab_i} + w_{at} \cdot A_{at_i} +
  w_{po} \cdot A_{po_i}
\end{equation}

We determine weights via the Analytical Hierarchy Process(AHP) [Saaty
1982]. We build a $4 \times 4$ reciprocal matrix by pair comparison:
\begin{center}
\begin{tabular}{|c|c|c|c|c|}
\hline
       &$A_{ac}$      &$A_{ab}$  &$A_{at}$    &$A_{po}$  \\ \hline
 $A_{ac}$ & 1           & 5 & 2            &1            \\ \hline
 $A_{ab}$ &$\frac{1}{5}$ & 1 &$\frac{1}{3}$ & $\frac{1}{4}$\\ \hline
 $A_{at}$ &$\frac{1}{2}$ & 3 & 1            &1            \\ \hline
 $A_{po}$ &1            & 4 & 1            &1            \\ \hline
\end{tabular}
\end{center}

The meaning of the number in each cell is explained in []. The numbers
themselves are based on our own subjective decisions.

We then get the weight of each factor by calculating the bigest
eigenvalue and it's corresponding eigenvector, as given in Table.

\begin{center}
\begin{tabular}{c|cccc}
  Factor  &$A_{ac}$    &$A_{ab}$   &$A_{at}$    &$A_{po}$\\ \hline
  Weight  &0.3805  &0.0709  &0.2371  &0.3030\\
\end{tabular}
\end{center}

We test the consistency of the preferences for this instance of the
AHP. For good consistency [Alonso and Lamata 2006, 446 - 447]:

\begin{itemize}
\item The principal eigenvalue $\lambda_{max}$ of the matrix should be
  close to the number n of alternatives, here 4; we get $\lambda_{max} = 4.047$.
\item The consistency index $CI = (\lambda_{max}-n)/(n-1)$ should be close to 0; we get $CI = 0.0157$.
\item The consistency ratio $CR = CI/RI$ (where RI is the average
  value of CI for random matrices) should be less than 0.1; we get $CR
  = 0.0182$.
\end{itemize}

Hence, our decision method displays perfectly acceptable consistency
and the weights are reasonable.

\subsection{Social Network Model}
\label{sec:social-network-model}

The social network model contains a directed weighted graph $G(V,E)$
in which $V$ denote the employees and $E$ denote the connection
between employees. Since there are personnel changes, $G(V,E)$ will
change with time goes by. In order to simulate this situation, we use
$G_t(V_t,E_t)$ instead of $G(V,E)$ where $t$ is a discrete
variable. So $G_t(V_t,E_t)$ denote the social network in the $t$-th
month.

First, we explain the way we build edges of $G_t(V_t,E_t)$.

When $t = 0$, there are about $370 \times 85\%$ nodes (employees) in
$G_t$. We build edges between employees in the same division or
office, since employees in the same division or office certainly know
each other. So each division or office form a complete graph and
employees in the differient division or office don't know others,
which is impossible. To solve this problem, we build 10 edges for each
employee with employees in other divisions with equal
probability. Then we build the other edges with probability
$p = \frac{\left|N_{i,t} \cap N_{j,t}\right|}{\left|N_{i,t} \cup
    N_{j,t}\right|}$
which is called Jaccard similarity coefficient[Jaccard 1901].

When $t > 0$, there will be employees leaving or joining the
company. If an employee leave, all his edges with other employees with
be deleted. If an employee newly join the company, he will follow
steps which employees at $t = 0$ take. This is the dynamic process of
graph $G_t(V_t,E_t)$.

Let $s_{ij,t}$ denote the weight from $i$ to $j$ at time $t$. We have
these properties of $G_t(V_t,E_t)$:

\begin{itemize}
\item $s_{ij,t} \ne s_{ij,t}$

  We made this graph directed and weighted because one person may
  consider another person his best friend while that person doesn't
  consider him a good friend. This situation may appear because of the
  relationship between superior and subordinate and the relationship
  between person with more friend and person with less friend. In
  general, $s_{ij,t} \ne s_{ij,t}$ for the reason above.

\item $s_{ij,t}=\dfrac{a_{ij,t}+f_{ij,t}}{2}$

  $a_{ij,t}$ denote the influence caused by the relationship between
  superior and subordinate, $f_{ij,t}$ denote the influence caused by
  the amount of friends for $i$ and $j$. We define
$$a_{ij,t}=\begin{cases}
  \frac{1}{2+\left|L_{i,t}-L_{j,t}\right|}, & L_{i,t} \ge L_{j,t} \\
  1-\frac{1}{2+\left|L_{i,t}-L_{j,t}\right|}, & L_{i,t} < L_{j,t} \\
\end{cases}$$
, where $L_{i,t}$ denote the level of $i$ at time $t$, and
$$f_{ij,t}=\frac{\left|N_{i,t} \cap
    N_{j,t}\right|}{\left|N_{i,t}\right|}$$.

\end{itemize}

Finally, we explain the concept of clustering coefficient [Duncan
J. Watts and Steven Strogatz 1998] of $i$ denoted as $c_{i,t}$, which
is a measure of the degree to which nodes in a graph tend to cluster
together. $c_{i,t}$ is defined as:

$$c_{i,t}=\dfrac{\left| \left\{ s_{jk,t}:v_j,v_k \in M_i,s_{jk,t} \in E
    \right\} \right|}{k_{i,t}(k_{i,t}-1)}$$
where
$M_{i,t}=\left\{ v_j:s_{ij,t} \in E \and s_{ji,t} \in E \right\}$ and
$k_{i,t}=\left| N_{i,t} \right|$.

\subsection{Promote and Churn Model}
\label{sec:promote-and-churn-model}

The main part of our model is the algorithm which controls the
behavior of employees. According to assumption 1 and assumption 2, we
have the methodology as follow:

\begin{itemize}
\item [\textit{step 1}]
\end{itemize}

The network of the company we've built changes in terms of
orgnizational churn and promotion. Considering various of factors in
reality, we build an orgnizational churn and promotion model to
predict the dynamic process.

The first part of the model is churn model.We define the churn rate of
an employee $i$ as $l_i$ to evaluate the probability to churn. $l_i$ is
usually controlled by sorts of factors to different degrees. We
divides $l_i$ into three parts:$l_{i1}$, $l_{i2}$ and$l_{i3}$.  $l_{i1}$
represents the churn rate because of lacking of promotion
opportunity. $l_{i2}$ represents the churn rate because of the changes
of other employees related to employee $i$. To simplify our model, we
presume that $l_i1$, $l_i2$ is linear correlated with $(1-p_i)$ and
$s_i$, which means

\begin{equation}
  l_{i1}=\lambda_1(1-p_i),l_{i2}=\lambda_2s_i
\end{equation}

$\lambda_1$ and $\lambda_2$ could be ensured in later calculation.

$l_{i3}$ represents the other factors we can't get any information from
the known conditons so that we regard it as stable.  Thus

\begin{equation}
  l_i = \lambda_1(1-p_i) + \lambda_2s_i +l_i3
\end{equation}

After analyzing a great deal of churn rate reports, we get the
composition of the three parts are 10.9\%, 2.2\% and 75.4\%
respectively. According to the percentage and the general
churn rate 18\%, we can calculate $\lambda_1$, $\lambda_2$ and
$l_{i3}$. As an orinigal condition,it satisfies

\begin{equation}
\begin{cases}
  \lambda_1 \sum_{i=1}^{370}(1-p_i)=10.9\% \times 370 \times 1.5\% \\
  \lambda_2 \sum_{i=1}^{370}s_i=2.2\% \times 370 \times 1.5\% \\
  l_i3=1.5 \% \times 75.4 \%
\end{cases}
\end{equation}

Thus we can use equation (4) to calculate the churn rate $l_i$.

The second part of the model is promotion model aimed to predict the
promotion condition.We define the promotion rate of an employee i as
$p_i$ to evaluate the probability to promote.As a matter of fact,if
there is a vacancy, judging if an employee suits the site involves
work experience and ability.It is essential that he is supposed to
have several years of work experience according to issue 6.If an
employee satisfies the experience condition,it turns out to think
about his ability.Since in people model,each employee's ability is
evaluated by a parameter $A_{D_i}$. For each level of position,it has
an ability standard,as shows in Figure[]. The ability of an employee
are supposed to reach the four standard parameters
respectively,otherwise its $p_i$ is 0.For those who reach the
standard, the promotion probability can be calculated by the equation:
$p_i =\dfrac{A_{D_i}}{\sum_{\alpha} A_{D_\alpha}}$ where $\alpha$ is
employee who have probability to promote.

\section{Performance and Analysis}
\label{sec:performance-and-analysis}

\subsection{Analysis for Task 2}
\label{sec:analysis-for-task-2}

a

\subsection{Analysis for Task 3}
\label{sec:analysis-for-task-3}

a

\subsection{Analysis for Task 4}
\label{sec:analysis-for-task-4}

a

\subsection{Analysis for Task 5}
\label{sec:analysis-for-task-5}

a

\section{Advice for HR}
\label{sec:advice-for-hr}

a

\subsection{Incentive Mechanism}
\label{sec:incentive-machanism}

a

\subsection{Matching Employees to the Right Position}
\label{sec:matching-employees-to-the-right-position}

a

\section{Team Science}
\label{sec:team-science}

a

\section{Sensitivity Analysis}
\label{sec:sensitivity-analysis}

a

\section{Strengths and Weaknesses}
\label{sec:strengths-and-weaknesses}

\subsection*{Strengths}
\label{sec:strengths}

\begin{itemize}
\item Our model make fully use of the theory of multilayer networks so
  that it quantizes the relation accurately and reasonablly.
\item Our model exellently proves the interaction among these
  factors:leave probability, promotion probability and productivity.
\item The network we built include both microcosmic part and
  macrocosmic part, and they react to each other.
\item Our model proves the effection of time.
\end{itemize}

\subsection*{Weaknesses}
\label{sec:weaknesses}

\begin{itemize}
\item Limited by the time,we neglected sorts of factors which are not so significant.In fact, the model still has space to be perfected.
\item The result has some randomness.
\end{itemize}

\section{Conclusions}
\label{sec:conclusions}

a

% the reference
\begin{thebibliography}{99}

\bibitem{}

\end{thebibliography}

\label{LastPage}

\end{document}

%%% Local Variables:
%%% mode: latex
%%% TeX-master: t
%%% End:
